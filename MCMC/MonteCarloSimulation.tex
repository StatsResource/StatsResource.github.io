\section*{What is Monte Carlo Simulation?}
Monte Carlo simulation, or probability simulation, is a technique used to understand the impact of risk
and uncertainty in financial, project management, cost, and other forecasting models.
\subsection*{Uncertainty in Forecasting Models}
When you develop a forecasting model – any model that plans ahead for the future – you make certain
assumptions. These might be assumptions about the investment return on a portfolio, the cost of a
construction project, or how long it will take to complete a certain task. Because these are projections
into the future, the best you can do is estimate the expected value.
You can't know with certainty what the actual value will be, but based on historical data, or expertise in
the field, or past experience, you can draw an estimate. While this estimate is useful for developing a
model, it contains some inherent uncertainty and risk, because it's an estimate of an unknown value.
\subsection*{Estimating Ranges of Values}
In some cases, it's possible to estimate a range of values. In a construction project, you might estimate
the time it will take to complete a particular job; based on some expert knowledge, you can also
estimate the absolute maximum time it might take, in the worst possible case, and the absolute
minimum time, in the best possible case. The same could be done for project costs. In a financial
market, you might know the distribution of possible values through the mean and standard deviation of
returns.

By using a range of possible values, instead of a single guess, you can create a more realistic picture of
what might happen in the future. When a model is based on ranges of estimates, the output of the
model will also be a range. 

This is different from a normal forecasting model, in which you start with some fixed estimates – say
the time it will take to complete each of three parts of a project – and end up with another value – the
total time for the project. If the same model were based on ranges of estimates for each of the three
parts of the project, the result would be a range of times it might take to complete the project. When
each part has a minimum and maximum estimate, we can use those values to estimate the total
minimum and maximum time for the project.
