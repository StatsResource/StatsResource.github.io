MathsCast Series C : Probability

What is a sample space?
What is a random experiment?
 
Simple example 1 : A Coin is tossed one time
Simple example 2 : A Coin is tossed twice
Simple example 3 : A dice is thrown once
Simple example 4 : A Dice is thrown twice  
MathsCast C2 : Union, Intersection and Complement
Venn Diagrams
P(AB)  : Probability of A union B


MA4004 Spring 2009

Spring 2009 Q3a


Spring 2009 Question 3b

99% Confidence Interval


Degrees of freedom ()
Degrees of freedom depends on the sample size (n).

Sample size (n)

If the sample size is less than or equal to 30, it is considered a small sample.(= n-1)
If the sample size is larger than 30, it is considered a large sample.(=)

Significance ()

Number of tails (k)

Hypothesis tests can either be one tailed or two tailed.
Confidence intervals are always two tailed procedures.


Murdoch Barnes table 4

Column (/k )
Row : = n-1(small samples) or(large samples)


Spring 2009 Question 4a

Spring 2009 Question 4b - Inference



Spring 2009 Question 5a - Chi Square Testing





Spring 2009 Question 5b - Simple Linear Regression



Categorical vs. Quantitative Variables
Discrete vs. Continuous Variables
Univariate vs. Bivariate Data
Sample Question on introductory Statistics
Normal Distribution Example 1
Normal Distribution Example 2
Standard Normal Distribution Table
The Normal Distribution as a Model for Measurements

Normal distribution
Normal distributions are a family of distributions that have the same general shape. They are symmetric with scores more concentrated in the middle than in the tails. Normal distributions are sometimes described as bell shaped. 

Examples of normal distributions are shown below. Notice that they differ in how spread out they are. The area under each curve is the same. The height of a normal distribution can be specified mathematically in terms of two parameters: the mean(μ) and the standard deviation (σ). 




Categorical vs. Quantitative Variables
Variables can be classified as categorical (aka, qualitative) or quantitative (aka, numerical).
Categorical. Categorical variables take on values that are names or labels. The color of a ball (e.g., red, green, blue) or the breed of a dog (e.g., collie, shepherd, terrier) would be examples of categorical variables. 
Quantitative. Quantitative variables are numerical. They represent a measurable quantity. For example, when we speak of the population of a city, we are talking about the number of people in the city - a measurable attribute of the city. Therefore, population would be a quantitative variable.
In algebraic equations, quantitative variables are represented by symbols (e.g., x, y, or z).
Discrete vs. Continuous Variables
Quantitative variables can be further classified as discrete or continuous. If a variable can take on any value between its minimum value and its maximum value, it is called a continuous variable; otherwise, it is called a discrete variable.
Some examples will clarify the difference between discrete and continouous variables.
Suppose the fire department mandates that all fire fighters must weigh between 150 and 250 pounds. The weight of a fire fighter would be an example of a continuous variable; since a fire fighter's weight could take on any value between 150 and 250 pounds. 
Suppose we flip a coin and count the number of heads. The number of heads could be any integer value between 0 and plus infinity. However, it could not be any number between 0 and plus infinity. We could not, for example, get 2.3 heads. Therefore, the number of heads must be a discrete variable.
Univariate vs. Bivariate Data
Statistical data is often classified according to the number of variables being studied.
Univariate data. When we conduct a study that looks at only one variable, we say that we are working with univariate data. Suppose, for example, that we conducted a survey to estimate the average weight of high school students. Since we are only working with one variable (weight), we would be working with univariate data.
Bivariate data. When we conduct a study that examines the relationship between two variables, we are working with bivariate data. Suppose we conducted a study to see if there were a relationship between the height and weight of high school students. Since we are working with two variables (height and weight), we would be working with bivariate data.
Sample Question on introductory Statistics

Problem 1
Which of the following statements are true?
I. All variables can be classified as quantitative or categorical variables. 
II. Categorical variables can be continuous variables. 
III. Quantitative variables can be discrete variables.
(A) I only 
(B) II only 
(C) III only 
(D) I and II 
(E) I and III
Solution
The correct answer is (E). All variables can be classified as quantitative or categorical variables. Discrete variables are indeed a category of quantitative variables. Categorical variables, however, are not numeric. Therefore, they cannot be classified as continuous variables.

Normal Distribution Example 1

An average light bulb manufactured by the Acme Corporation lasts 300 days with a standard deviation of 50 days. Assuming that bulb life is normally distributed, what is the probability that an Acme light bulb will last at most 365 days?
The value of the normal random variable is 365 days.
The mean is equal to 300 days.
The standard deviation is equal to 50 days.




Solution: Given a mean score of 300 days and a standard deviation of 50 days, we want to find the cumulative probability that bulb life is less than or equal to 365 days. Thus, we know the following:

We enter these values into the Normal Distribution Calculator and compute the cumulative probability. The answer is: P( X < 365) = 0.90. Hence, there is a 90% chance that a light bulb will burn out within 365 days.


Standard Normal Distribution Table
A standard normal distribution table shows a cumulative probability associated with a particular z-score. Table rows show the whole number and tenths place of the z-score. Table columns show the hundredths place. The cumulative probability (often from minus infinity to the z-score) appears in the cell of the table.
For example, a section of the standard normal table is reproduced below. To find the cumulative probability of a z-score equal to -1.31, cross-reference the row of the table containing -1.3 with the column containing 0.01. The table shows that the probability that a standard normal random variable will be less than -1.31 is 0.0951; that is, P(Z < -1.31) = 0.0951.

z
0.00
0.01
0.02
0.03
0.04
0.05
0.06
0.07
0.08
0.09
-3.0
0.0013
0.0013
0.0013
0.0012
0.0012
0.0011
0.0011
0.0011
0.0010
0.0010
...
...
...
...
...
...
...
...
...
...
...
-1.4
0.0808
0.0793
0.0778
0.0764
0.0749
0.0735
0.0722
0.0708
0.0694
0.0681
-1.3
0.0968
0.0951
0.0934
0.0918
0.0901
0.0885
0.0869
0.0853
0.0838
0.0823
-1.2
0.1151
0.1131
0.1112
0.1093
0.1075
0.1056
0.1038
0.1020
0.1003
0.0985
...
...
...
...
...
...
...
...
...
...
...
3.0
0.9987
0.9987
0.9987
0.9988
0.9988
0.9989
0.9989
0.9989
0.9990
0.9990

Of course, you may not be interested in the probability that a standard normal random variable falls between minus infinity and a given value. You may want to know the probability that it lies between a given value and plus infinity. Or you may want to know the probability that a standard normal random variable lies between two given values. These probabilities are easy to compute from a normal distribution table. Here's how.
Find P(Z > a). The probability that a standard normal random variable (z) is greater than a given value (a) is easy to find. The table shows the P(Z < a). The P(Z > a) = 1 - P(Z < a). 

Suppose, for example, that we want to know the probability that a z-score will be greater than 3.00. From the table (see above), we find that P(Z < 3.00) = 0.9987. Therefore, P(Z > 3.00) = 1 - P(Z < 3.00) = 1 - 0.9987 = 0.0013. 
Find P(a < Z < b). The probability that a standard normal random variables lies between two values is also easy to find. The P(a < Z < b) = P(Z < b) - P(Z < a). 

For example, suppose we want to know the probability that a z-score will be greater than -1.40 and less than -1.20. From the table (see above), we find that P(Z < -1.20) = 0.1151; and P(Z < -1.40) = 0.0808. Therefore, P(-1.40 < Z < -1.20) = P(Z < -1.20) - P(Z < -1.40) = 0.1151 - 0.0808 = 0.0343.
In school or on the Advanced Placement Statistics Exam, you may be called upon to use or interpret standard normal distribution tables. Standard normal tables are commonly found in appendices of most statistics texts.
The Normal Distribution as a Model for Measurements
Often, phenomena in the real world follow a normal (or near-normal) distribution. This allows researchers to use the normal distribution as a model for assessing probabilities associated with real-world phenomena. Typically, the analysis involves two steps.
Transform raw data. Usually, the raw data are not in the form of z-scores. They need to be transformed into z-scores, using the transformation equation presented earlier: z = (X - μ) / σ.
Find probability. Once the data have been transformed into z-scores, you can use standard normal distribution tables, online calculators (e.g., Stat Trek's free normal distribution calculator), or handheld graphing calculators to find probabilities associated with the z-scores.
The problem in the next section demonstrates the use of the normal distribution as a model for measurement.



 
 
%================================================================================================================================================%
Introductory Probability Notes
 

Elementary properties of Probability
Geometric Distribution
 
Elementary properties of Probability
 
  1. P(A) = 1 - P(A)
2. P() = 0
3. P(A)P(B)ifAB
 
4. P(A) \leq 1
5.
P ( AandB) = P(A) +P(B) - P( AorB )
 
6. If A_1, A_2, \ldots A_3  are n arebitary events in S then
 
%----------------------------------------------------------------------%
 
There are 3 persons in a room
What is the probability that two people have the same birthday?
For the sake of simplicity, exclude leap years.
Outcomes =\{ (Jan 1, Jan 1, Jan 1), ( Jan 1, Jan 1, Jan 2), \ldots , (Dec 31, Dec 31, Dec 31) \}
Number of outcomes = (365)^3.
 
%----------------------------------------------------------------------%
Geometric Distribution
 
Conside the experiment of tossing a fair coin repeatedly and counting th number of tosses required until the first head appears.
 
%----------------------------------------------------------------------%
 
A binary source generates digits 1 and 0 with probability  0.6 and 0.4 respecitvely.
 
What is the probability that two ones and three zeroes will occur in a five digit sequence?
 
 
P(X = 2) =52(0.6)2(0.4)3= 200.350.0064 = 0.23
 
A fair coin is flipped 10 times. Find the probability of 2,3 or 4 heads
 
P(X=2)
P(X=3)
p(X=4)
 

 
%--------------------------------------------------------------------------------------%

Sample Space: This is the set of all possible outcomes of a random experiment, and is denoted $S$. An element in $S$ is called a sample point.
Each outcome of a random experiment corresponds to a sample point.
Find the sample space for the experiment of tossing a coin (a) one and (b) twice.
(a) There are two possible outcomes: heads or tails.
Thus
S= {H,T}where $H$ and $T$ represent head and tail respectively.
(b) There are four possible outcomes, which are pairs of heads and tails.
Thus S= {HH,HT,TH,TT} 
%--------------------------------------------------------------------------------------%
Considering a coin toss experiment. A fair coin is tossed three times.
Let X be the random variable that counts the number of heads in each sample point.
P(X\leq 1)
P(X\>1)
p(0<X<3)
Set of Eligible outcomes = \{1,2\}

%--------------------------------------------------------------------------------------%

A production line produces 1000-ohm resistors that have 10\% tolerance.
Let $X$ denote the resistance of a resistor.
Assuming that X is a normally distributed random variable with mean 1000 and variance 2500, find the probability that a randomly selected resistor will be rejected.
Solution:
Let A be the event that a resistor is rejected.
P(A)  = P(X \leq 900) + P( X \geq 1100)
P(X \leq 900)
Z value
Z_{900} = { 900 - 1000\over 50} = -2
X_{1100} = { 1100 - 1000\over 50} = 2
Probability of being rejected is 0.045 (i.e. 4.5 \%)

%--------------------------------------------------------------------------------------%
An information source generates symbols at random from a four letter alphabet {A,C,G,T} with the following probabilities
P(a) = {1 \over 2}
P(c) = {1 \over 4}
P(g) = {1 \over 8}
P(t) = {1 \over 8}
a coding scheme encodes these symbols into binary code as follows
a 0
c 10
g 110
t 111
Let X be the random variable denoting the lenght of the code, i.e. the number of bits.
(a) What is the range of X?
The outcome with the shortest possible length is $AAAA$, which contains 4 bits.
The outcomes with the longest possible lenght would contain 12 bits.
 

%--------------------------------------------------------------------------------------%
A Bernouilli experiment is a random experiment, the outcome of which can be classified in but one of two mutually exclusive and exhaustive ways, say "success" or "failure".

A sequence of Bernouillli trials occurs when a Bernouilli experiment is repeated several times, and the outcome of each successive experiment is independent of the last.

The distribution function F_X(x) = P(X \leq x) where $-< x$
%--------------------------------------------------------------------------------------%
Discrete Random Variables
Let X be a random variable with a finite of countably infinite number of points.
%--------------------------------------------------------------------------------------%

Uniform distribution
A random variable is called a uniform random variable the interval (a,b) if its probability density function is given

f_X(x) = \begin{cases}  {1 \over b-a } \\ 0 \end{cases}
%--------------------------------------------------------------------------------------%



MathsCast C3 : Conditional Probability
    P(A | B)  : Probability of A given that event B has happened.
 
MathsCast C4 : Bayes Theorem
MathsCast C5 : Random Variable
What is a random variable? 
 
 
A. Continuous Interest

A principal of €3,500 is invested with interest compounded continuously.
What is the required interest rate if the investment is to grow to \euro 6,250 after 5
years?

B) Histograms

(4 marks) Comment on the shape of the histogram. Based on the shape of the
histogram, what is the best measure of centrality and variability

Two types of histogram

Skewed
Symmetric

Median and Interquartile Range
Mean and Standard deviation ( or Variance)


C) Sampling without replacement

A lot of semiconductor chips contains 20 that are defective.

Two chips are selected at random, without replacement from the lot.


What is the probability that the first one is defective?

What is the probability that the second one is defective given that the first one was defective?

Independent events

What is the probability that the second one is defective given that the first chip was not defective?

What is the probability that both chips are defective.


H) Effective Interest Rates

100 is subject to interest at 10% compounded quarterly. What is the accrued amount after one year?



%==================================================================%
Section G : counting Problems
The Choose operator

{n \choose k}  = {n! \over k! (n-k)!} 

%==================================================================%
\section*{Section H Introduction to Descriptive Statistics}
G) Computing the Median of a data set
Odd size or even size data set

F) Characteristics of Data sets

Range max and min
in Order?

J) Interquartile Range

%==================================================================%
Section J Bivariate Data 

Correlation


correlation coefficient must be between -1 and +1.

also it has no units. Also it is not a proportion, so it shouldnt be expreszed as a percentage.


Slope / Intercept  Models 

%==================================================================================================================================================================================%
Normal Distribution
1 Worked Examples
Question 4.
Question 5.
Normal Distribution
Example

A standard Z table can be used to find probabilities for any normal curve problem that has been converted to Z scores. For the table, refer to the text. The Z distribution is a normal distribution with a mean of 0 and a standard deviation of 1.
The following steps are helpfull when working with the normal curve problems:
1. Graph the normal distribution, and shade the area related to the probability you want to find.
2. Convert the boundaries of the shaded area from X values to the standard normal random variable Z values using the Z formula above.
3. Use the standard Z table to find the probabilities or the areas related to the Z values in step 2.
 
Problem One:
 
A survey showed that salaries for accountants with six to nine years experience are normally distributed with a mean of € 48,000 and a standard deviation of € 4000. What is the probability that an accountant earns:
 
More than €52,000?                                                            
More than €45,500?
Between €45,500 and €52,000?   

1 Worked Examples
 
Problem Two:

Graduate Management Aptitude Test (GMAT) scores are widely used by graduate schools of business as an entrance requirement. Suppose that in one particular year, the mean score for the GMAT was 476, with a standard deviation of 107. Assuming that the GMAT scores are normally distributed, answer the following questions:The following figure shows a graphic representation of this problem.




Question 1. What is the probability that a randomly selected score from this GMAT falls between 476 and 650?

 
Applying the Z equation, we get: Z = (650 - 476)/107 = 1.62. The Z value of 1.62 indicates that the GMAT score of 650 is 1.62 standard deviation above the mean. The standard normal table gives the probability of value falling between 650 and the mean. The whole number and tenths place portion of the Z score appear in the first column of the table. Across the top of the table are the values of the hundredths place portion of the Z score. Thus the answer is that 0.4474 or 44.74% of the scores on the GMAT fall between a score of 650 and 476.

Problem Three
 
What is the probability of receiving a score greater than 750 on a GMAT test that has a mean of 476 and a standard deviation of 107?
 
 
 

 
This problem is asking for determining the area of the upper tail of the distribution. The Z score is: Z = ( 750 - 476)/107 = 2.56. From the table, the probability for this Z score is 0.4948. This is the probability of a GMAT with a score between 476 and 750. The rule is that when we want to find the probability in either tail, we must substract the table value from 0.50. Thus, the answer to this problem is: 0.5 - 0.4948 = 0.0052 or 0.52%. Note that P(X >= 750) is the same as P(X >750), because, in continuous distribution, the area under an exact number such as X=750 is zero. The following figure shows a graphic representation of this problem.
 
Question 4.
What is the probability of receiving a score of 540 or less on a GMAT test that has a mean of 476 and a standard deviation of 107? i.e., P(X <= 540)="?." we are asked to determine the area under the curve for all values less than or equal to 540. the z score is: z="(540" 476)/107="0.6." from the table, the probability for this z score is 0.2257 which is the probability of getting a score between the mean (476) and 540. the rule is that when we want to find the probability between two values of x on either side of the mean, we just add the two areas together. Thus, the answer to this problem is: 0.5 + 0.2257 = 0.73 or 73%. The following figure shows a graphic representation of this problem.

Figure 6
Question 5.
 What is the probability of receiving a score between 440 and 330 on a GMAT test that has a mean of 476 and a standard deviation of 107? i.e., P(330 < 440)="?." the solution to this problem involves determining the area of the shaded slice in the lower half of the curve in the following figure.

Figure 7
In this problem, the two values fall on the same side of the mean. The Z scores are: Z1 = (330 - 476)/107 = -1.36, and Z2 = (440 - 476)/107 = -0.34. The probability associated with Z = -1.36 is 0.4131, and the probability associated with Z = -0.34 is 0.1331. The rule is that when we want to find the probability between two values of X on one side of the mean, we just subtract the smaller area from the larger area to get the probability between the two values. Thus, the answer to this problem is: 0.4131 - 0.1331 = 0.28 or 28%.

Example Two:

Suppose that a tire factory wants to set a mileage guarantee on its new model called LA 50 tire. Life tests indicated that the mean mileage is 47,900, and standard deviation of the normally distributed distribution of mileage is 2,050 miles. The factory wants to set the guaranteed mileage so that no more than 5% of the tires will have to be replaced. What guaranteed mileage should the factory announce? i.e., P(X <= ?)="5%.<br"> In this problem, the mean and standard deviation are given, but X and Z are unknown. The problem is to solve for an X value that has 5% or 0.05 of the X values less than that value. If 0.05 of the values are less than X, then 0.45 lie between X and the mean (0.5 - 0.05), see the following graph.

Figure 8

Refer to the standard normal distribution table and search the body of the table for 0.45. Since the exact number is not found in the table, search for the closest number to 0.45. There are two values equidistant from 0.45-- 0.4505 and 0.4495. Move to the left from these values, and read the Z scores in the margin, which are: 1.65 and 1.64. Take the average of these two Z scores, i.e., (1.65 + 1.64)/2 = 1.645. Plug this number and the values of the mean and the standard deviation into the Z equation, you get:
Z =(X - mean)/standard deviation or -1.645 =(X - 47,900)/2,050 = 44,528 miles.
Thus, the factory should set the guaranteed mileage at 44,528 miles if the objective is not to replace more than 5% of the tires.

The Normal Approximation to the Binomial Distribution:

In lecture note number 5 we talked about the binomial probability distribution, which is a discrete distribution. You remember that we said as sample sizes get larger, binomial distribution approach the normal distribution in shape regardless of the value of p (probability of success). For large sample values, the binomial distribution is cumbersome to analyze without a computer. Fortunately, the normal distribution is a good approximation for binomial distribution problems for large values of n. The commonly accepted guidelines for using the normal approximation to the binomial probability distribution is when (n x p) and [n(1 - p)] are both greater than 5.

Example:

Suppose that the management of a restaurant claimed that 70% of their customers returned for another meal. In a week in which 80 new (first-time) customers dined at the restaurant, what is the probability that 60 or more of the customers will return for another meal?, ie., P(X >= 60) =?.

The solution to this problem can can be illustrated as follows:
First, the two guidelines that (n x p) and [n(1 - p)] should be greater than 5 are satisfied: (n x p) = (80 x 0.70) = 56 > 5, and [n(1 - p)] = 80(1 - 0.70) = 24 > 5. Second, we need to find the mean and the standard deviation of the binomial distribution. The mean is equal to (n x p) = (80 x 0.70) = 56 and standard deviation is square root of [(n x p)(1 - p)], i.e., square root of 16.8, which is equal to 4.0988. Using the Z equation we get, Z = (X - mean)/standard deviation = (59.5 - 56)/4.0988 = 0.85. From the table, the probability for this Z score is 0.3023 which is the probability between the mean (56) and 60. We must substract this table value 0.3023 from 0.5 in order to get the answer, i.e., P(X >= 60) = 0.5 -0.3023 = 0.1977. Therefore, the probability is 19.77% that 60 or more of the 80 first-time customers will return to the restaurant for another meal. See the following graph.

Figure 9


%===================================================================%
% Section K Computing Confidence Intervals

D) Confidence interval for a mean

\bar{x}

Quantile

s sample standard deviation

sample size 

standard error  \frac{s}{\sqrt{n}}



E) Quantiles computing Confidence intervals

Is the variance known?
Is the sample a large sample or small sample

is n \leq 30 ? small
is n > 30 large


Confidence intervals
point estimate 
sample mean
sample proportion
difference of two sample means
hypothesis testing
1 Write out null and alternative hypothesis
2 Compute test statistic 
	First need to compute the standard error.
3 Determine the quantile for critical value
4 Decision rule 
Important Considerations
Is the sample a small sample or large sample?
	Threshold n=30
		Small sample - use the student t distribution
		Large sample - use standard normal distribution
What is the significance level ? 
Is the procedure one tailed or two tailed? (k =1 or 2)

Standard Error p(100-p)`n





computing a v
Threshold value
alue based on a probability.
D
first we are hiven a probability vslue

then we find the z value
the 
n we use the standardisation formula to compute the probability.



the standard deviation is computed as follows.




 

Approach to Normal Distribution Questions
1)Given the mean (μ) and standard deviation (σ), and a particular value (Xo)
2)Use the standardisation formula to determine Zo
3)Use the table to find P(Z ≥ Zo)
4)This value is equivalent to P(X ≥ Xo)
 
Population of widgets are known to have weight
 
and standard deviation
 Questions
 
1) Find the population of widgets greater than 150 kilograms
2) Find the proportion of widgets less than 130 Kilograms

